\documentclass[a4paper, 11pt]{article}

\usepackage[utf8]{inputenc}

\begin{document}

Le terme de {\em système de recommandation} regroupe l'ensemble des tâches d'apprentissage automatique ({\em machine learning}) où l'on veut faire correpondre les goûts d'utilisateurs et de produits selon leurs goûts et leurs propriétés. 
Les méthodes de l'état-de-l'art reposent typiquement sur des méthodes de factorisation de matrices, reconnues pour leur efficacité.

Récemment, l'explicabilité et la justifiabilité des systèmes d'apprentissage automatique en général, et des systèmes de recommandations en particulier, ont retenu l'attention de la communauté scientifique.
Cependant, ces questionnements se heurtent à de nombreux problèmes; l'un d'entre eux est que l'équité et la diversité ne sont pas formellement définis; un travail fondamental est nécessaire.

Les graphes, c'est-à-dire des ensembles de n\oe{}uds reliés entre eux par des arêtes, sont un modèle naturel pour les systèmes de recommandation.
On peut recourir à un graphe dirigé, pondéré, dynamique, biparti (ou une combinaison de ces propriétés), selon l'application considérée.
Une propriété clef de ces graphes est leur interprétabilité.

\section{Déroulement du stage et gratification}

Le stage se déroulera entre le laboratoire Cedric du Conservatoire National des Arts et Métiers (3 mois), à Paris, et le Riken Advanced Intelligence Project (3 mois), à Tokyo.

Une gratification de stage de 577,05 euros est prévue pour la partie en France, ainsi que la moitié du coût de transport.

Une gratification de stage de 66000 yens (environ 530 euros), ainsi qu'une prise en charge du logement (dans la limite de 100000 yens -- environ 850 euros par mois) et la totalité du transport est prévue. Le billet d'avion aller/retour pour Tokyo sera également pris en charge.

\section{Contact}

Vous pouvez envoyer votre demande accompagnée d'un CV aux adresses \texttt{fournier@cnam.fr} et \texttt{tiphaine.viard@riken.jp}.

\end{document}
